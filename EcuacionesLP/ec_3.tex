%---
%title: 'Modelo completo de Laura-Petrarch'
%label: 'Cauchy'
%reference: 'articulo1'
%...

%\documentclass{article}

%\begin{document}

\begin{eqnarray*}
  \frac{dL(t)}{dt} &= -\alpha_{L}L(t) + \beta_{L}P\left(1-\left(\frac{P}{\gamma_{L}}\right)^{2}\right) + \beta_{L}(A_{P}) \\
  \frac{dP(t)}{dt} &= -\alpha_{P}P(t) + \beta_{P}L(t) + \beta_{P}\frac{A_{L}}{1+\delta_{P}I_{P}(t)} \\
  \frac{dI_{P}(t)}{dt} &= -\alpha_{IP}I_{P}(t) + \beta_{IP}P(t)
\end{eqnarray*}

%\end{document}

%El **\left(** al principio y al final **\right)** sirve para hacer los paréntesis
%tan grandes como la expresión matemática lo requiera
